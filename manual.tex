\documentclass[a4paper,12pt,titlepage]{scrartcl}

%Pakete
%\usepackage[left=3cm,right=2cm,top=2cm,bottom=3cm]{geometry} %Seitenränder
\usepackage[utf8]{inputenc} % ermöglicht die direkte Eingabe der Umlaute 
\usepackage [german]{babel} %Spracheinstellungen

\usepackage {csquotes} % Anführungszeichen nach dem Stil \enquote{ich bin zitiert.}
\usepackage{subscript} % erlaubt Tiefstellen von Zahlen und Text
\sloppy %macht ungefähren Blocksatz, wenn nichts anderes an Trennhilfen was nützt

% Hurenkinder und Schusterjungen verhindern
\clubpenalty10000
\widowpenalty10000
\displaywidowpenalty=10000

\usepackage[final]{graphicx}

\usepackage [
citestyle = authoryear,
bibstyle = philosophy-classic,% Zitierstil
isbn=false,                % ISBN nicht anzeigen, %gleiches geht mit nahezu allen anderen Feldern
doi=false,
pagetracker=true,          % ebd. bei wiederholten ngaben (false=ausgeschaltet, page=Seite, spread=Doppelseite, true=automatisch)
%ümaxbibnames=3,            % maximale Namen, die im Literaturverzeichnis angezeigt werden 
maxcitenames=3,            % maximale Namen, die im Text angezeigt werden, ab 4 wird u.a. nach den ersten Autor angezeigt
autocite=inline,           % regelt Aussehen für \autocite (inline=\parancite)
block=space,               % kleiner horizontaler Platz zwischen den Feldern
backref=false,              % Seiten anzeigen, auf denen die Referenz vorkommt
backrefstyle=three+,       % fasst Seiten zusammen, z.B. S. 2f, 6ff, 7-10
date=short                % Datumsformat
]{biblatex}
\newcommand{\todo}[1]{\marginpar{#1}}
\bibliography{sources}

\DeclareBibliographyDriver{report}{%
  \usebibmacro{bibindex}%
  \usebibmacro{begentry}%
  \usebibmacro{author}%
  \setunit{\labelnamepunct}\newblock
  \usebibmacro{title}%
  \newunit
  \printlist{language}%
  \newunit\newblock
  \usebibmacro{byauthor}%
  \newunit\newblock
%  \printfield{type}%
%  \setunit*{\addspace}%
  \printfield{number}%
  \newunit\newblock
  \printfield{version}%
  \newunit
  \printfield{note}%
  \newunit\newblock
  \usebibmacro{institution+location+date}%
  \newunit\newblock
  \usebibmacro{chapter+pages}%
  \newunit
  \printfield{pagetotal}%
  \newunit\newblock
  \iftoggle{bbx:isbn}
    {\printfield{isrn}}
    {}%
  \newunit\newblock
  \usebibmacro{doi+eprint+url}%
  \newunit\newblock
 \usebibmacro{addendum+pubstate}%
  \setunit{\bibpagerefpunct}\newblock
  \usebibmacro{pageref}%
  \usebibmacro{finentry}}
\usepackage[colorlinks=true,linkcolor=black,citecolor=black,urlcolor=black,breaklinks=true]{hyperref}
%\usepackage[style=authortitle-icomp]{biblatex}

%\bibliography{Pfad/zur/Bibliographie-Datei/Dateiname} 


\title{User Manual}
\subtitle{For the FINken Quadrocopters}
\author{Sebastian Mai}
\date{\today} % sollte man u.U. anpassen ;)


\begin{document}
\maketitle
\tableofcontents
\section{Hardware}

\subsection{Akkus und Ladegerät}
Die Akkus die wir benutzen sind 2-Zellen Litium-Polymer-Akkus mit 950mAh Kapazität.
Zum Laden müssen die Akkus sowohl mit beiden Steckern mit dem Ladegerät verbunden werden, danach kann man mit dem Balance-Programm die Akkus laden.
Wichtig ist, dass im Ladegerät 2S ausgewählt ist und die Stromstärke nicht über 2A eingestellt wird.
(mit 1A Stromstärke zu laden ist schonender für die Akkus, über 4A können die Akkus anfangen zu brennen)
Die Akkus sollten nicht ohne Aufsicht geladen werden.

Am Quadrocopter kann man den Akku mit dem JST-Connector anschließen, der 4-Polige Balancestecker sollte ebenfalls angeschlossen werden, damit die Spannung im Flug überwacht werden kann (Polarität beachten!).

Um neue Firmware zu flashen muss der Quadrocopter über USB mit dem Rechner verbunden werden, der Akku sollte dabei nicht angesteckt sein.

\subsection{Komponenten}
\subsubsection{Aktuatoren}

Die Motoren des Quadrocopters (und letzendlich auch die Propeller) werden von sogenannten Speedcontrollern gesteuert, die aus dem Gleichstrom der Batterie einen 30-Phasen Wechseltstrom machen der die Motoren antreibt.
Als Steuereingang verwenden die Speedcontroller ein PWM-Signal, dass vom FlightControler erzeugt wird.


\subsubsection{Flight Control}

Als FlightControl-Board verwenden wir ein Board vom Typ Paparazzi-Lia, auf dem bereits ein 10-DOM Lagesensor (IMU) integriert ist und auf dem die Paparazzi-Firmware läuft. Hier werden sowohl die Low-Level-Lageregelung als auch logisch höhere Funktionen wie Navigation umgesetzt und alle Aktuatoren und Sensoren angeschlossen.

\subsubsection{Sonar}
Die Sonarsensoren sind neben der IMU die wichtigsten Sensoren. Die Lagesensoren sind über I2C an die Flightcontrol angebunden.
Die Adressen sind 0x71 bis 0x75, wobei 0x71 die Adresse des vorderen Sensors ist und im Uhrzeigersinn nummeriert wird.
Der untere Sensor hat die Adresse 0x75.

Der Code für die Sensortreiber befindet sich in sw/airborne/modules/sonar/sonar\_array.c.

\subsubsection{Farbsensor}
Der Farbsensor ist genau wie die Sonarsensoren per I$^2$C angebunden. Ein Treiber muss noch geschrieben werden\footnote{Zur Orientierung kann der Treiber für die Sonarsensoren genutzt werden.}.


\subsubsection{Bluetooth}
Bluetooth wird zum Empfang der Telemetriedaten und zur Steuerung des Autopiloten\footnote{Es geht als nicht um Manuellen Flug, sondern um Kommandos wie "Starten, Landen, Wegpunkt anfliegen etc.} genutzt.
Um Bluetooth nutzen zu können muss der Quadrocopter mit dem Rechner mit der Bodenstation gepairt werden – das ist leider bei jedem Start von neuem Notwendig, obwohl der Bluetoothstandart auch automatisches Pairing vorsieht.
\emph{Die PIN zum Pairing ist 1234.}


\subsection{Fernbedienung}
Die Fernbedienung dient zur manuellen Steuerung des Quadrocopters und zum umschalten der Autopiloten-Modi.

\subsubsection{Manuelle Steuerung}
Der Wichtigste Schalter ist open Rechts: Der Kill-Schalter. Dieser Schalter funktioniert quasi als Not-Aus – alle Motoren werden sofort ausgeschalten.
Zum losfliegen muss dieser Schalter deaktiviert werden.
Wichtig sind außerdem die beiden Gimbles der Fernbedienung: Links ist Schub und Yaw, rechts Pitch und Roll.

\subsubsection{Autopiloten-Modi}

\section{Software}

\subsection{Paparazzi-Center}

Das Paparazzi Center bietet bequeme Möglichkeiten zum Kompilieren und Flashen der Firmware und zum starten der zum Flug benötigten Tools.
Um die Anwendung auszuführen muss im Paparazzi-Verzeichnis der Befehl \enquote{ ./paparazzi } eingegeben werden.
Wenn Änderungen an der Konfiguration für das Bodensegment gemacht werden muss das make-Target \enquote{gcs} neu kompiliert werden.


Wenn man mit dem Quadrocopter fliegen will sollte man in etwa folgendes tun:
\begin{enumerate}
	\item Upload der neuen Firmware (falls neue Firmware geschrieben wurde)
	\item Trennen von USB und verbinden der Batterie und Akkuüberwachung
	\item Herstellen der Bluetooth-Verbindung (über das Betriebssystem)
	\item Start einer Session (Button oben Rechts)
	\item Fliegen
	\item Kill-Switch auf \enquote{Kill}, abstecken der Batterie (und Akkuüberwachung)
\end{enumerate}

\subsubsection{Groundstation}
Die Groundstation dient zum steuern des Quadrocopters – hier sieht man diverse Informationen wie aktueller Batteriestand, Navigationsmodus ...
Von hier können auch Kommandos wie \enquote{Takeoff}, \enquote{Land}, etc. gegeben werden.

\subsubsection{Data-Link}
Die Groundstation bekommt ihre Informationen über Bluetooth 4.0. Da in BT4.0 keine Serielle Verbindung im Protokoll enthalten ist, muss diese Verbindung zuerst über ein externes Programm hergestellt werden. Danach kann sich das Data-Link-Tool von Paparazzi auf das neu erstellte Pseudoterminal \path{/dev/pts/X} verbinden.

Dazu muss in \path{/home/finken/bt-bridge/mkpty.sh} die MAC-Adresse des aktuellen Bluetoothmoduls einkommentiert werden.
Danach kann das entsprechende Device (der Pfad von mkpty.sh angezeigt) in den enstprechenden Parameter des Data-Link-Programmes im Paparazzi-Center eingetragen werden.
Danach kann das Data-Link-Programm gestartet werden und die Groundstation erhällt alle Informationen über den Quadrocopter.

\subsubsection{Messages}
Unter tools $\Rightarrow$ messages befindet sich ein nützliches Werkzeug mit dem sich die Telemetriedaten ausgeben lassen – unter anderem auch die ausgabe der Sonar-Sensoren.
Die angezeigten Daten können mit dem Real-Time-Plotter auch visualisiert werden.



\subsection{Code}

\subsubsection*{Versionsverwaltung}
Der Code wird aktuell verwaltet mit der Versionskontrollsoftware "git". 
Eine gute Referenz für die Benutzung von git findet man unter \url{http://www.ndpsoftware.com/git-cheatsheet.html}.
Der Code wird gehostet unter \url{https://github.com/ovgu-FINken/paparazzi} und kann heruntergeladen werden mit \enquote{git clone}\footnote{Da mit git mehrere Features parallel entwickelt werden können, muss u.U. noch ein Branch ausgecheckt werden}.

\subsubsection*{Organisation}

Der Code im Paparazzi-Repository teilt sich für uns in zwei Bereiche ein: Einerseits die Konfigurationsdatein in \path{conf} und andererseits der Quelltext für die Firmware unter \path{sw/airborne}.
Idealerweise sollte der Coder für die Firmware also für mehrere verschiedene Konfigurationen funktionieren, so dass z.B. statt einem Quadrocopter auch ein Flugzeug eingesetzt werden kann.


\subsubsection*{Airframe}
Im Airframe sind alle Spezifika des Fluggerätes zusammengefasst – also alles von der Drehrichtung der Motoren bis zur Adressierung einzelner Sensoren.
Änderungen müssen hier eigentlich nur gemacht werden, sofern Hardware neu angebunden wird oder die Hardware selbst verändert wird.
Unter anderem sind hier auch die PID-Werte für die Lageregelung festgelegt, die angepasst werden sollten wenn sich das Gewicht des Quadrocopters ändert.
Unser Airframefile befindet sich unter \path{conf/airframes/ovgu/quadrotor_lia_ovgu.xml}.


\subsubsection*{Flightplan}
Im Flightplan(\path{conf/flight_plans/ovgu.xml} wird quasi eine "Mission" für den Quadrocopter festgelegt.
Paparazzi sieht dafür im wesentlichen Wegpunktnavigation vor, aber unter anderem können hier "Exceptions" definiert werden, so das der Quadrocopter vor einer Wand ausweicht, falls er im normalen Flugmodus sonst dagegenstoßen würde.
Momentan wird die eigentliche Steuerung von einer Funktion im Sonartreiber ausgeführt, die über den Flightplan aufgerufen wird.

\subsection{Modell}

Um Strukturierter Arbeiten zu können, folgt die Software dem folgenden Modell der Copter.

\subsubsection{Sensormodell}
Diverse Variablen lassen sich durch Sensorik und Virtuelle Sensoren (also z.B. Sensorfusion oder Filter) beobachten.
Dazu gehören: Geschwindigkeiten und Beschleunigungen in allen Achsen außer der Höhe, die Absolute Höhe, die Messungen der Sonarsensoren und noch einige weitere Werte.

\subsubsection{Aktuatoren}
Das Aktuatormodell ist beschränkt auf die Angabe einer Fluglage, die durch die Lageregelung des Quadrocopters umgesetzt werden soll. Dazu gehören Pitch- und Roll-Winkel, die Ableitung des Yaw-Winkels und der Schub.

\subsubsection{Systemmodell}
Der Zustand des Quadrocopters lässt sich beschreiben aus $\dot x$, $\dot y$, dem Yaw-Winkel($\phi$), dem Aktuellen Schub ($t$) und der aktuellen Höhe.
Der Zusammenhang zwischen Aktuatoren und $\dot x$ bzw. $\dot y$ kann dabei ungefähr beschrieben werden wie folgt: \\
\begin{math}
	\dot x = k_t \cdot t \sin \alpha \\ 
	\dot y = k_t \cdot t \sin \beta \\
\end{math}
$k_t$ ist dabei eine Konstante die sich aus der Masse des Quadrocopters und dem Verhältnis von vorgegebenen Schub und der tatsächlich erzeugten Kraft ergibt.

\subsubsection{Umweltmodell}
Die Umwelt des Quadrocopters soll zu nächst durch ein sehr einfaches Modell beschrieben werden: Durch den Abstand und Winkel des nächstgelegenen Objektes.

\newpage
\sloppy
\printbibliography
\end{document}
